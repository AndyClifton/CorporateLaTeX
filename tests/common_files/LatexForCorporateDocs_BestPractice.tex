\emph{Create a structure before you get too far.} Authors will find it easier to write documents and make changes if they separate the content of the document from the structure.
\begin{enumerate}
\item Each new LaTeX document should be placed in it's own directory. 
\item Create a main LaTeX file that just contains the preamble, custom commands and uses \texttt{input} to call the content. See Section \ref{sec:FileStructure} for an example where each \texttt{chapter} is contained in its own file. In an article, each \texttt{section} could be contained in its own file.
\item Keep the number of packages used to a minimum. Not all packages can be used as they lack compatibility.
\end{enumerate}

\emph{Focus on content, not appearance.} Don't spend hours trying to adjust fonts, headers or spacing between lines. 
\begin{enumerate}
\item Don't throw in lots of \texttt{clearpage}s or other commands to push material around. LaTeX is designed to handle that. 
\item Resist the temptation to add or subtract space, change lengths or do other things to modify the layout. 
\item Write!
\end{enumerate}