PDF tagging is a process whereby the components of the PDF document (headings, figures, tables, text) are marked so that a document reader can understand the document. This is useful when text to speech converters are being used. The process of tagging is also known as structuring, so that a tagged document might also be referred to as a structured document\footnote{This is a test}.

At this time tags cannot be added reliably within LaTeX. Instead, they should be added after the PDF is compiled using a PDF editor such as Adobe's Acrobat Pro. 
