Class files control the formatting and presentation of documents. The class files currently available include:
\begin{description}
\item[CorporateReport.cls]{compiles the document using the LaTeX \emph{report} class, with corporate formatting. This is intended for longer documents and allows the use of chapters.}
\item[CorporateArticle.cls] compiles the document using the LaTeX \emph{article} class, with corporate formatting. This is intended for shorter documents such as journal articles. This class does not support the use of chapters.
\item[CorporateCompact.cls] is like \emph{CorporateArticle}, but with reduced spacing.
\end{description}

\emph{resources/CorporateResources.tex} contains the common packages and formatting descriptions that are implemented by the \emph{Corporate*.cls} classes.

As with normal classes, options are passed to the class in the \verb+\documentclass+ line:

\begin{lstlisting}
\documentclass[option 1, ..., option n]{CorporateArticle}
\end{lstlisting}

All of the usual options can be used with the \emph{Corporate*.cls} classes, including \emph{twocolumn, letterpaper,} and so-on.

Options specific to \emph{Corporate*.cls} include:
\begin{description}
\item[draft]{add a `draft' watermark to all pages.}
\item[blacklinks]{make all links the same color as the rest of the body text.}
\item[logo]{add the logo to all pages.}
\item[tagged]{used PDF tagging}
\end{description}

The \emph{Corporate*.cls} files call a variety of other packages. Packages are codes that modify the appearance or behaviour of LaTeX to achieve something. Table \ref{Tab:Packages} lists the packages that are explicitly called by \emph{Corporate*.cls} or \emph{CorporateResources.tex} in the order they are called in. These packages often call other packages, so this is not an exhaustive list.

\begin{table*}[!ht]
\centering
\caption[Packages loaded by the Corporate classes]{Packages loaded by the Corporate classes, in alphabetical order.}
\label{Tab:Packages}
\begin{tabular}[h]{l p{0.6\textwidth}}
\toprule
Package & Function\\
\midrule
accessibility & generates the PDF document structure and tagging \\
amsfonts, amssymb& supplies AMS fonts, which are useful for mathematics \\
babel & activates language-appropriate hyphenation rules\\
booktabs & improves the formatting of tables \\
caption & required to generate captions for floats\\
fontenc & enables direct typing of international characters \\
geometry & sets page size and margins \\
graphicx & graphics handling, including \emph{.eps} figures \\
hyphenat & improves spacing and breaking of hyphenated words \\
listings & enables the inclusion of high-quality computer code listings\\
mathptmx & changes fonts \\
nag & checks that packages are up to date and looks for bad habits in LaTeX code\\
opensans & sets Google's \emph{Open Sans} as the default font\\
parskip & required for better spacing\\
pdfcomment & required for tool-tips. Also calls the \emph{hyperref} package.\\
setspace & required for better spacing\\
subcaption & provides the \texttt{subfigure} environment to produce sub figures \\
tocloft & improved table of contents and list of figures/tables in memoir documents\\
tocbibind & Adds bibliography, index, and contents entries to the Table of Contents in memoir documents\\
todonotes & inline and margin to-do notes \\
xcolor & Driver-independent color extensions for LaTeX and pdfLaTeX\\
\bottomrule
\end{tabular}
\end{table*}

It should be noted that the `english` option to Babel really means American english.
