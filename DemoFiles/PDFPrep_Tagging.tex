PDF tagging is a process whereby the components of the PDF document (headings, figures, tables, text) are marked so that a document reader can understand the document. This is useful when text to speech converters are being used. The process of tagging is also known as structuring, so that a tagged document might also be referred to as a structured document\footnote{This is a test}.

The \emph{accessibility} package can be used to add tags to a LaTeX document. This package is called by choosing the \emph{tagged} option when calling the \emph{Corporate*.cls} document class:

\begin{lstlisting}[language={[LaTeX]Tex}]
\documentclass[option 1, ..., tagged]{CorporateArticle}
\end{lstlisting}

Tags do not always work reliably. It may be easier to add them after the PDF is compiled using a PDF editor such as Adobe's Acrobat Pro. 
